%!TEX root = ../Libro.tex
\chapter{Revisión de otros LMS}
Para realizar la selección de funcionalidades que se podrían incorporar a Ósmosis2 se estudiaron varios sistemas de gestión del aprendizaje. A continuación se muestran las características de cada LMS y las semejanzas entre ellos \label{Comparativa de LMS} \citep{EduTools}.
\section{Sistemas revisados}
\subsection{Sakai 2.3 }
	\paragraph{Enfoque Pedagógico:} Constructivismo, cognitivismo.
	\paragraph{Características:} 
	\begin{itemize}
		\item \textbf{Foros:}
			\begin{itemize}
				\item Permite habilitar/deshabilitar el recibir los posts como e-mails.
				\item Herramienta de revisión ortográfica.
				\item Los foros pueden hacerse visibles o bloquearse (ver y no participar) en una fecha específica
				\item Los estudiantes pueden ser moderadores de foros
				\item Se pueden marcar las discusiones como ``leídas''
			\end{itemize}
	\end{itemize}
	\begin{itemize}
		\item \textbf{Diario online/notas:}
		\begin{itemize}
			\item Los estudiantes pueden agregar material (aprobado por el profesor) de utilidad en la herramienta ``Recursos''.
			\item Se muestran los usuarios conectados que están usando el sistema.
		\end{itemize}
	\end{itemize}
	\begin{itemize}
		\item \textbf{Portafolios:}
		\begin{itemize}	
			\item OSP (Open Source Portfolio), herramienta provisional que ofrece las funcionalidades de un portafolio.
		\end{itemize}
	\end{itemize}
	\begin{itemize}
		\item \textbf{Pizarra:}
		\begin{itemize}	
			\item Se puede incorporar Elluminate, Breeze, entre otros.
		\end{itemize}
	\end{itemize}
	\begin{itemize}
		\item \textbf{Intercambio de Archivos:}
			\begin{itemize}
				\item Entrega de tareas a través de ``drop boxes'' o casilleros del profesor.
			\end{itemize}
	\end{itemize}
	\begin{itemize}
		\item \textbf{Correo Interno:}
			\begin{itemize}
				\item Envío de mensajes a toda la clase
			\end{itemize}
	\end{itemize}
	\begin{itemize}
		\item \textbf{Chat:}	
			\begin{itemize}
				\item El sistema crea archivos de registro para todas las salas de chat.
			\end{itemize}
	\end{itemize}
	\begin{itemize}
		\item \textbf{Calendario/Agenda:}
			\begin{itemize}
				\item Los profesores y los estudiantes pueden añadir eventos el el calendario del curso.
				\item El estudiante posee una página de inicio personal con los cursos en los cuales participa, todos los eventos o actividades de su calendario personal.
				\item El estudiante puede ver la calificación de las actividades realizadas, el total de puntos posibles, calificación del curso y comparar sus calificaciones con las del salón.
			\end{itemize}
	\end{itemize}
	\begin{itemize}
		\item \textbf{Comunidad:}
			\begin{itemize}
				\item Sakai Wiki Tool: Permite a los usuarios  crear, compartir y administar contenido en un ambiente de Wiki. Se pueden compartir conceptos con otros Wikis open sourc como Wikipedia, TWiki, phpWiki, etc.
				\item News/RSS - para visualizar el contenido de fuentes online (RSS feeds)
		\end{itemize}
	\end{itemize}
	\begin{itemize}
		\item \textbf{Capacidades de búsqueda:}
			\begin{itemize}
				\item El estudiante puede realizar búsquedas en todo el cursos.
				\item El estudiante puede buscar en todos los hilos de discusión.--
			\end{itemize}
	\end{itemize}
	\begin{itemize}
		\item \textbf{Ayuda y Orientación del curso:}
			\begin{itemize}
				\item El sistema incluye tutoriales en línea para ayudar al estudiante a aprender el uso del sistema
				\item Los estudiantes pueden acceder a un menú de ayuda para cualquiera de las herramientas del sistema.
			\end{itemize}
	\end{itemize}
	\begin{itemize}
		\item \textbf{Grupos de Trabajo:}
			\begin{itemize}
				\item El profesor puede asignar estudiantes a grupos.
				\item A cada grupo se le pueden asignar tareas o actividades específicas.
				\item Los grupos pueden ser privados o monitoreados por el profesor.
			\end{itemize}
	\end{itemize}
	\begin{itemize}
		\item \textbf{Tipos de pruebas:}
			\begin{itemize}
				\item Tipos de pruebas: selección múltiple, respuestas múltiples, completar el espacio en blanco,  preguntas cortas, encuestas, ensayos, matching, calculadas, preguntas que contengan imagenes, videos , audio,etc.
				\item Manejo automatizado de Pruebas
				\item El sistema puede aleatorizar las preguntas  y respuestas.
				\item El profesor puede crear exámenes de prueba.
				\item El profesor puede definir tiempos límites para las evaluaciones.
				\item El profesor puede permitir múltiples intentos.
				\item Los estudiantes puede revisar intentos anteriores de una prueba.
				\item Los profesores definen si se debe mostrar las respuestas correctas.
				\item Soporte para pruebas automatizadas
				\item Los profesores pueden crear un banco personal de pruebas.
				\item El sistema provee análisis de los datos de las pruebas.
				\item Se pueden importar preguntas desde bancos de pruebas que soporten QTI.
				\item Herramientas de puntuación online
				\item Se puede puntuar cada estudiante en todas las preguntas o cada pregunta de todos los estudiantes.
				\item Control de calificaciones del curso (gradebooks):
				\item Cuando el profesor agrega una actividad al curso, ésta se agrega automáticamente a la tabla de actividades a evaluar.
				\item Se puede crear una escala de evaluación que contemple porcentajes, estatus de aprobado/reprobado, calificación alfabética.
				\item El profesor puede agregar calificaciones para actividades off-line.
				\item Se pueden exportar las calificaciones contenidas en la tabla.
				\item Los profesores pueden evaluar las respuestas de los estudiantes de forma anónima.
			\end{itemize}
	\end{itemize}
	\begin{itemize}
		\item \textbf{Seguimiento del estudiante:}
			\begin{itemize}
				\item Site Statistics tool: agrupa y muestra datos estadísticos acerca de la clase, como o un todo o por cada estudiante, lo que han hecho y los sitios que han accesado.
			\end{itemize}
	\end{itemize}
	\begin{itemize}
		\item \textbf{Manejo del Curso:}
			\begin{itemize}
				\item El instructor puede programar la aparición de eventos, tareas o pruebas, anuncios en el sistema de acuerdo a una fecha de inicio y una fecha de fin.
			\end{itemize}
	\end{itemize}
	\begin{itemize}
		\item \textbf{Plantillas del curso:}
			\begin{itemize}
				\item El sistema permite a los administradores usar un curso existente o una plantilla predefinida como base para un nuevo curso.
				\item El contenido del curso puede ser cargado a través de WebDAV.
			\end{itemize}
	\end{itemize}
	\begin{itemize}
		\item \textbf{Personalización de la interfaz:}
			\begin{itemize}
				\item El sistema proporciona plantillas (por defecto) para la interfaz de los cursos.
				\item Los profesores pueden modificar los iconos y colores de un curso.
			\end{itemize}
	\end{itemize}
	\begin{itemize}
		\item \textbf{Herramientas de diseño de cursos:}
			\begin{itemize}
				\item Los profesores pueden organizar los learning objects, las herramientas del curso y el contenido del material que puede ser reusable.
				\item Reutilización de los cursos como plantilla para otros cursos en los cuales se maneje la misma información.
			\end{itemize}
	\end{itemize}
	\begin{itemize}
	\item \textbf{Compartir/Reusar contenido:}
		\begin{itemize}
			\item El material puede ser importado o copiado del sitio de un curso a otro.
		\end{itemize}
	\end{itemize}
	\begin{itemize}
		\item \textbf{Requerimientos de base de datos:}
			\begin{itemize}	
				\item El sistema soporta MySQL.
				\item El sistema soporta Oracle.
				\item La aplicación requiere sólo una base de datos y puede coexistir con tables de otras aplicaciones.
			\end{itemize}
	\end{itemize}
	\begin{itemize}
		\item \textbf{Comunidad de Usuarios}
	\end{itemize}

\subsection{Claroline 1.8.1}
	\paragraph{Enfoque Pedagógico:} Cognitivista, constructivista.
	\paragraph{Características:} 
	\begin{itemize}
		\item \textbf {Foros:}
			\begin{itemize}
				\item Permite habilitar/deshabilitar el recibir los posts como e-mails.
				\item Herramienta de revisión ortográfica
			\end{itemize}
	\end{itemize}
	\begin{itemize}
		\item \textbf{Manejo de discusiones:}
			\begin{itemize}
				\item Los profesores pueden permitir a los estudiantes la creción de grupos de discusión.
			\end{itemize}
	\end{itemize}
	\begin{itemize}
		\item \textbf{Intercambio de Archivos:}
			\begin{itemize}
				\item Entrega de tareas a través de ``drop boxes'' o casilleros del profesor.
				\item Posibilidad de compartir archivos
				\item Notificación de cambios en el material a través de RSS.
			\end{itemize}
	\end{itemize}
	\begin{itemize}
		\item \textbf{Correo Interno:}
			\begin{itemize}
				\item Envío de mensajes a toda la clase.
			\end{itemize}
	\end{itemize}
	\begin{itemize}
		\item \textbf{Chat:}
			\begin{itemize}
				\item Soporte ilimitado de grupos de discusión simultáneos.
				\item El sistema crea archivos de registro para todas las salas de chat.
			\end{itemize}
	\end{itemize}
	\begin{itemize}
		\item \textbf {Calendario/Agenda:}
			\begin{itemize}
				\item El estudiante posee una página de inicio personal con los cursos en los cuales participa, todos los eventos o actividades de su calendario personal.
			\end{itemize}
	\end{itemize}
	\begin{itemize}
		\item \textbf{Capacidades de búsqueda:}
			\begin{itemize}
				\item El estudiante puede buscar en todos los hilos de discusión.
			\end{itemize}
	\end{itemize}
	\begin{itemize}
		\item \textbf{Grupos de Trabajo:}
			\begin{itemize}
				\item El profesor puede asignar estudiantes a grupos.
				\item El sistema puede crear grupos, de forma aleatoria y con un cierto tamaño.
				\item A cada grupo se le pueden asignar tareas o actividades específicas.
				\item Los grupos pueden ser privados o monitoreados por el profesor.
				\item Los estudiantes escoger o seleccionar grupos.
			\end{itemize}
	\end{itemize}
	\begin{itemize}
		\item \textbf{Tipos de pruebas:}
			\begin{itemize}
				\item Tipos de pruebas: selección múltiple, respuestas múltiples, completar el espacio en blanco,  preguntas que contengan imagenes, videos , audio,etc.
				\item Manejo automatizado de Pruebas
				\item El sistema puede aleatorizar las preguntas  y respuestas.
				\item El profesor puede crear exámenes de prueba.
				\item El profesor puede definir tiempos límites para las evaluaciones.
				\item El profesor puede permitir múltiples intentos.
				\item Los estudiantes puede revisar intentos anteriores de una prueba.
				\item Los profesores definen si se debe mostrar las respuestas correctas.
				\item Soporte para pruebas automatizadas
				\item Los profesores pueden crear un banco personal de pruebas.
				\item El sistema provee análisis de los datos de las pruebas.
				\item Se pueden importar preguntas desde bancos de pruebas que soporten QTI.
			\end{itemize}
	\end{itemize}
	\begin{itemize}
		\item \textbf{Seguimiento del estudiante:}
			\begin{itemize}
				\item Historial de navegacion de cada estudiante.
				\item Los profesores pueden obtener reportes del número de veces, tiempo, fecha, frecuencia y dirección IP de cada estudiante que ha accesado al contenido de un curso, foro, tareas, etc.
				\item Estadísticas de lo que más hace un estudiante/ todos los estudiantes.
			\end{itemize}
	\end{itemize}
	\begin{itemize}
		\item \textbf{Personalización de la interfaz:}
			\begin{itemize}
				\item Las instituciones pueden crear su propio estilo y plantillas para todo el sistema, incluyendo logo, encabezados y pie de páginas.
			\end{itemize}
	\end{itemize}
	\begin{itemize}
		\item \textbf{Requerimientos de base de datos:}
			\begin{itemize}
				\item El sistema soporta MySQL.
				\item La aplicación requiere sólo una base de datos y puede coexistir con tables de otras aplicaciones.
			\end{itemize}
	\end{itemize}
	\begin{itemize}
		\item \textbf{Comunidad de Usuarios:}\\
		\citep{CLA_COM2008}
			\begin{itemize}
				\item 93 países, entre los cuáles se encuentran: Argentina, Australia, Brasil, Canadá, Colombia, Francia, Alemania, Israel, Mexico, entre otros.
				\item 1184 organizaciones.
			\end{itemize}
	\end{itemize}	

\subsection{ATutor 1.5.4}
	\paragraph{Enfoque Pedagógico:} Cognitivista, constructivista.
	\paragraph{Características:}
	\begin{itemize}
		\item \textbf{Foros:}
		\begin{itemize}
			\item Permite habilitar/deshabilitar el recibir los posts como e-mails.
			\item Los estudiantes pueden ser asignados como administradores de foros.
		\end{itemize}
	\end{itemize}
	\begin{itemize}
		\item \textbf{Manejo de discusiones:}
		\begin{itemize}
			\item Las discusiones pueden ser compartidas entre cursos, departamentos y cualquier otra unidad de la institución.
		\end{itemize}
	\end{itemize}
	\begin{itemize}
		\item \textbf{Intercambio de Archivos:}
			\begin{itemize}
				\item Entrega de tareas a través de ``drop boxes'' o casilleros del profesor
				\item Administradores pueden definir el tamaño de espacio en disco para cada usuario
				\item Los estudiantes tienen una carpeta privada en la cual pueden almacenar sus archivos.
			\end{itemize}
	\end{itemize}
	\begin{itemize}
		\item \textbf{Correo Interno:}
			\begin{itemize}
				\item Envío de mensajes a toda la clase
				\item Libreta de direcciones
				\item Capacidad de hacer forward a correo extrerno.
			\end{itemize}
	\end{itemize}
	\begin{itemize}
		\item \textbf{Chat:}
			\begin{itemize}
				\item Soporte ilimitado de grupos de discusión simultáneos.
				\item El sistema crea archivos de registro para todas las salas de chat.
				\item Los estudiantes pueden crear nuevas salas de chat.
			\end{itemize}
	\end{itemize}
	\begin{itemize}
		\item \textbf{Pizarra:}
			\begin{itemize}
				\item La pizarra soporta imágenes y archivos PowerPoint.
			\end{itemize}
	\end{itemize}
	\begin{itemize}
		\item \textbf{Calendario/Agenda:}
			\begin{itemize}
				\item Los profesores y los estudiantes pueden añadir eventos el el calendario del curso.
				\item El estudiante posee una página de inicio personal con los cursos en los cuales participa, todos los eventos o actividades de su calendario personal.
			\end{itemize}
	\end{itemize}
	\begin{itemize}
		\item \textbf{Capacidades de búsqueda:}
			\begin{itemize}
				\item El estudiante puede realizar búsquedas en todo el curso.
			\end{itemize}
	\end{itemize}
	\begin{itemize}
		\item \textbf{Trabajo desconectado:}
			\begin{itemize}
				\item Los estudiantes pueden ``compilar'' y descargar todo el contenido de un curso para ser impreso o guardado localmente.
			\end{itemize}
	\end{itemize}
	\begin{itemize}
		\item \textbf{Ayuda y Orientación del curso:}
			\begin{itemize}
				\item Los estudiantes pueden acceder a un menú de ayuda para cualquiera de las herramientas del sistema.
				\item Los estudiantes y profesores pueden contribuir con el material de ayuda
				\item El material de ayuda es buscable.
			\end{itemize}
	\end{itemize}
	\begin{itemize}
		\item \textbf{Grupos de Trabajo:}
			\begin{itemize}
				\item El profesor puede asignar estudiantes a grupos.
				\item El sistema puede crear grupos, de forma aleatoria y con un cierto tamaño.
				\item Cada grupo puede tener su propio foro de discusión.
				\item A cada grupo se le pueden asignar tareas o actividades específicas.
				\item Los grupos pueden ser privados o monitoreados por el profesor.
			\end{itemize}
	\end{itemize}
	\begin{itemize}
		\item \textbf{Comunidad:}
			\begin{itemize}
				\item Los estudiantes, de distintos cursos, pueden participar en todas las salas de chat o foros del sistema.
				\item Se pueden asignar tests/quizzes a los grupos.
			\end{itemize}
	\end{itemize}
	\begin{itemize}
		\item \textbf{Tipos de pruebas:}
			\begin{itemize}
				\item Tipos de pruebas: selección múltiple, respuestas múltiples, completar el espacio en blanco, preguntas cortas, encuestas, ensayos, preguntas que contengan imagenes, videos , audio,etc.
				\item Manejo automatizado de Pruebas
				\item El sistema puede aleatorizar las preguntas  y respuestas.
				\item El profesor puede crear exámenes de prueba.
				\item El profesor puede definir tiempos límites para las evaluaciones.
				\item El profesor puede permitir múltiples intentos.
				\item Los estudiantes puede revisar intentos anteriores de una prueba.
				\item Los profesores definen si se debe mostrar las respuestas correctas.
			\end{itemize}
	\end{itemize}
	\begin{itemize}
		\item \textbf{Soporte para pruebas automatizadas:}
			\begin{itemize}
				\item Los profesores pueden crear un banco personal de pruebas.
				\item El sistema provee análisis de los datos de las pruebas.
				\item Se pueden importar preguntas desde bancos de pruebas que soporten QTI.
			\end{itemize}
	\end{itemize}
	\begin{itemize}
		\item \textbf{Herramientas de puntuación online:}
		\begin{itemize}
			\item Se puede puntuar cada estudiante en todas las preguntas o cada pregunta de todos los estudiantes.
			\item Las puntuaciones pueden ser parciales para respuestas parcialmente correctas
		\end{itemize}
	\end{itemize}
	\begin{itemize}
		\item \textbf{Manejo automatizado de pruebas}
			\begin{itemize}
				\item Los resultados de las pruebas pueden permanecer ocultos hasta que todas las pruebas hayan sido recibidas.
				\item Los resultados de la prueba pueden ser liberados en el momento en que un estudiante entrega la prueba.
				\item El sistema puede mostrar resultados parciales de  las preguntas de selección y verdadero/falso.
			\end{itemize}
	\end{itemize}
	\begin{itemize}
		\item \textbf{Manejo del Curso:}
			\begin{itemize}
				\item El instructor puede programar la aparición de eventos, tareas o pruebas, anuncios en el sistema de acuerdo a una fecha de inicio y una fecha de fin.
				\item Los profesores pueden enlazar las discusiones a fechas específicas o eventos del curso.
			\end{itemize}
	\end{itemize}
	\begin{itemize}
		\item \textbf{Seguimiento del estudiante:}
			\begin{itemize}
				\item Historial de navegacion de cada estudiante.
			\end{itemize}
	\end{itemize}
	\begin{itemize}
		\item \textbf{Compartir/Reusar contenido:}
			\begin{itemize}
				\item Los profesores pueden compartir contenidos con otros profesores o estudiantes a través de un repositorio central de learning objects.
			\end{itemize}
	\end{itemize}
	\begin{itemize}
		\item \textbf{Plantillas del curso:}
			\begin{itemize}
				\item El sistema permite a los administradores usar un curso existente o una plantilla predefinida como base para un nuevo curso.
			\end{itemize}
	\end{itemize}
	\begin{itemize}
		\item \textbf{Personalización de la interfaz:}
			\begin{itemize}
				\item El profesor puede cambiar el orden y los nombres de los items ubicados en el menú del curso.
				\item El sistema puede soportar múltiples instituciones, departamentos, escuelas u otras organizaciones en una sola instalación donde cada unidad puede tener su estilo y sus plantillas así como imágenes, encabezados u pie de páginas.
			\end{itemize}
	\end{itemize}
	\begin{itemize}
		\item \textbf{Herramientas de diseño de cursos:}
			\begin{itemize}
				\item Reutilización de los cursos como plantilla para otros cursos en los cuales se maneje la misma información.
			\end{itemize}
	\end{itemize}
	\begin{itemize}
		\item \textbf{Requerimientos de base de datos:}
			\begin{itemize}
				\item El sistema soporta MySQL.
				\item La aplicación requiere sólo una base de datos y puede coexistir con tables de otras aplicaciones.
			\end{itemize}
	\end{itemize}
	\begin{itemize}
		\item \textbf{Comunidad de Usuarios}
	\end{itemize}
	

\subsection{Moodle 1.8}
	\paragraph{Enfoque Pedagógico:} Constructivismo social (Se puede adaptar para el uso cognitivista puro).
	\paragraph{Características:}
	\begin{itemize}
		\item \textbf{Foros:}
			\begin{itemize}
				\item Permite habilitar/deshabilitar el recibir los posts como e-mails.
				\item Herramienta de revisión ortográfica.
				\item Posibilidad de que los estudiantes sean administradores de foros individuales del curso
				\item RSS de los posts
			\end{itemize}
	\end{itemize}
	\begin{itemize}
		\item \textbf{Manejo de discusiones:}
			\begin{itemize}
				\item Los profesores pueden permitir a los estudiantes la creción de grupos de discusión.
				\item Las discusiones pueden ser compartidas entre cursos, departamentos y cualquier otra unidad de la institución.
				\item Se debe opinar en el foro como requisito para ver los posts del resto de los estudiantes que participan.
			\end{itemize}
	\end{itemize}
	\begin{itemize}
		\item \textbf{Intercambio de Archivos:}
			\begin{itemize}
				\item Entrega de tareas a través de ``drop boxes'' o casilleros del profesor.
			\end{itemize}
	\end{itemize}
	\begin{itemize}
		\item \textbf{Correo Interno:}
			\begin{itemize}
				\item Envío de mensajes a toda la clase.
				\item Libreta de direcciones.
				\item Capacidad de hacer forward a correo extrerno.
			\end{itemize}
	\end{itemize}
	\begin{itemize}
		\item \textbf{Chat:}
			\begin{itemize}
				\item Soporte ilimitado de grupos de discusión simultáneos.
				\item El sistema crea archivos de registro para todas las salas de chat.
				\item Los estudiantes pueden crear nuevas salas de chat.
			\end{itemize}
	\end{itemize}
	\begin{itemize}
		\item \textbf{Calendario/Agenda:}
			\begin{itemize}
				\item Los profesores y los estudiantes pueden añadir eventos el el calendario del curso.
				\item El estudiante posee una página de inicio personal con los cursos en los cuales participa, todos los eventos o actividades de su calendario personal.
				\item El estudiante puede ver la calificación de las actividades realizadas, el total de puntos posibles, calificación del curso y comparar sus calificaciones con las del salón.
			\end{itemize}
	\end{itemize}
	\begin{itemize}
		\item \textbf{Pizarra:}
			\begin{itemize}
				\item Se puede incorporar Elluminate, DimDim, etc.
			\end{itemize}
	\end{itemize}
	\begin{itemize}
		\item \textbf{Capacidades de búsqueda:}
			\begin{itemize}
				\item El estudiante puede buscar en todos los hilos de discusión.
			\end{itemize}
	\end{itemize}
	\begin{itemize}
		\item \textbf{Ayuda y Orientación del curso:}
			\begin{itemize}
				\item El sistema incluye tutoriales en línea para ayudar al estudiante a aprender el uso del sistema
				\item Los estudiantes pueden acceder a un menú de ayuda para cualquiera de las herramientas del sistema.
			\end{itemize}
	\end{itemize}
	\begin{itemize}
		\item \textbf{Grupos de Trabajo:}
			\begin{itemize}
				\item El profesor puede asignar estudiantes a grupos.
				\item Cada grupo puede tener su propio foro de discusión.
				\item Cada grupo puede tener chat o pizarra.
				\item Los estudiantes, de distintos cursos, pueden participar en todas las salas de chat o foros del sistema.
			\end{itemize}
	\end{itemize}
	\begin{itemize}
		\item \textbf{Portafolios:}
			\begin{itemize}
				\item Los estudiantes pueden crear una página principal para cada curso.
			\end{itemize}
	\end{itemize}
	\begin{itemize}
		\item \textbf{Tipos de prueba}
			\begin{itemize}
				\item Tipos de pruebas: selección múltiple, respuestas múltiples, completar el espacio en blanco, matching, ordenamientos, calculadas, encuestas, ensayos, ordenar una frase. Se pueden personalizar el tipo de preguntas a realizar, preguntas que contengan imagenes, videos , audio,etc..
			\end{itemize}
	\end{itemize}
	\begin{itemize}
		\item \textbf{Manejo automatizado de pruebas}
			\begin{itemize}
				\item El sistema puede aleatorizar las preguntas  y respuestas.
				\item El profesor puede crear exámenes de prueba.
				\item El profesor puede definir tiempos límites para las evaluaciones.
				\item El profesor puede permitir múltiples intentos.
				\item Los estudiantes puede revisar intentos anteriores de una prueba.
				\item Los profesores definen si se debe mostrar las respuestas correctas.
				\item El sistema soporta el Protocolo de Quices Remotos, que permite que las preguntas se muestren y sean evaluadas externamente por medio de servicios web.
		\end{itemize}
	\end{itemize}
	\begin{itemize}
		\item \textbf{Soporte para pruebas automatizadas:}
			\begin{itemize}
				\item Los profesores pueden crear un banco personal de pruebas.
				\item El sistema provee análisis de los datos de las pruebas.
				\item Se pueden importar preguntas desde bancos de pruebas que soporten QTI.
			\end{itemize}
	\end{itemize}
	\begin{itemize}
		\item \textbf{Herramientas de puntuación online:}
			\begin{itemize}
				\item Se puede puntuar cada estudiante en todas las preguntas o cada pregunta de todos los estudiantes.
			\end{itemize}
	\end{itemize}
	\begin{itemize}
		\item \textbf{Control de calificaciones del curso (gradebooks):}
			\begin{itemize}
				\item Cuando el profesor agrega una actividad al curso, ésta se agrega automáticamente a la tabla de actividades a evaluar.
				\item Se puede crear una escala de evaluación que contemple porcentajes, estatus de aprobado/reprobado, calificación alfabética.
				\item El profesor puede agregar calificaciones para actividades off-line.
				\item Se pueden exportar las calificaciones contenidas en la tabla.
			\end{itemize}
	\end{itemize}
	\begin{itemize}
		\item \textbf{Seguimiento del estudiante:}
			\begin{itemize}
				\item Historial de navegación de cada estudiante.
				\item Los profesores pueden obtener reportes del número de veces, tiempo, fecha, frecuencia y dirección IP de cada estudiante que ha accesado al contenido de un curso, foro, tareas, etc.
				\item Estadísticas de lo que más hace un estudiante/ todos los estudiantes.
			\end{itemize}
	\end{itemize}
	\begin{itemize}
		\item \textbf{Plantillas del curso:}
			\begin{itemize}
				\item El sistema permite a los administradores usar un curso existente o una plantilla predefinida como base para un nuevo curso.
				\item El contenido del curso puede ser cargado a través de WebDAV.
			\end{itemize}
	\end{itemize}
	\begin{itemize}
		\item \textbf{Personalización de la interfaz:}
			\begin{itemize}
				\item El sistema proporciona plantillas (por defecto) para la interfaz de los cursos.
				\item Los profesores pueden modificar los iconos y colores de un curso.
				\item El profesor puede cambiar el orden y los nombres de los items ubicados en el menú del curso.
				\item El sistema puede soportar múltiples instituciones, departamentos, escuelas u otras organizaciones en una sola instalación donde cada unidad puede tener su estilo y sus plantillas así como imágenes, encabezados u pie de páginas.
			\end{itemize}
	\end{itemize}
	\begin{itemize}
		\item \textbf{Herramientas de diseño de cursos:}
			\begin{itemize}
				\item Los profesores pueden organizar los learning objects, las herramientas del curso y el contenido del material que puede ser reusable.
				\item Reutilización de los cursos como plantilla para otros cursos en los cuales se maneje la misma información.
			\end{itemize}
	\end{itemize}
	\begin{itemize}
		\item \textbf{Compartir/Reusar contenido:}
			\begin{itemize}
				\item Los profesores pueden hacer una copia completa de todo el curos y/o items individuales del curso y compartirlos con otros profesores o cargarlos en uno de los sistemas eCMS con integración en Moodle (Hive, Odalis, etc.)
			\end{itemize}
	\end{itemize}
	\begin{itemize}
		\item \textbf{Requerimientos de base de datos:}
			\begin{itemize}
				\item El sistema soporta MySQL.
				\item El sistema soporta Oracle.
				\item El sistema soporta MS SQL Server
				\item La aplicación requiere sólo una base de datos y puede coexistir con tables de otras aplicaciones.
				\item El sistema soporta PostgreSQL.
			\end{itemize}
	\end{itemize}
	\begin{itemize}
		\item \textbf{Comunidad de Usuarios:} \\
		\citep{MOD_COM2008}\\
		199 países entre los cuales se encuentran: Argentina, Australia, Brasil, Canadá, Chile, China, Colombia, Ecuador, El Salvador, España, Estados Unidos, Japón, India, Italia, Irlanda, Israel, Mexico, Panamá, Paraguay, Perú, Venezuela, entre otros.
	\end{itemize}

\subsection[Blackboard LS 4.1]{Blackboard Learning System Vista 4.1 Enterprise License}
	\paragraph{Enfoque Pedagógico:} Cognitivismo, constructivismo.
	\paragraph{Características:}
	\begin{itemize}
		\item \textbf{Foros:}
			\begin{itemize}
				\item Herramienta de revisión ortográfica.
				\item Asociar una discusión al contenido de un curso.
			\end{itemize}
	\end{itemize}
	\begin{itemize}
		\item \textbf{Intercambio de Archivos:}
			\begin{itemize}
				\item Entrega de tareas a través de ``drop boxes'' o casilleros del profesor
				\item Administradores pueden definir el tamaño de espacio en disco para cada usuario
				\item Los estudiantes tienen una carpeta privada en la cual pueden almacenar sus archivos.--
			\end{itemize}
	\end{itemize}
	\begin{itemize}
		\item \textbf{Correo Interno:}
			\begin{itemize}
				\item Envío de mensajes a toda la clase.
				\item Libreta de direcciones.
				\item Capacidad de hacer forward a correo extrerno.
				\item Los estudiantes pueden colocar notas a cualquier página.
				\item Posibilidad de que los estudiantes agreguen al material del curso sus apuntes para crear una guía de estudio.--
			\end{itemize}
	\end{itemize}
	\begin{itemize}
		\item \textbf{Chat:}
			\begin{itemize}
				\item Soporte ilimitado de grupos de discusión simultáneos.
				\item El sistema crea archivos de registro para todas las salas de chat.
				\item Los estudiantes pueden ver quien más está conectado en el sistema e invitarlo a la sala de chats.
				\item El sistema NO limita el número de salas simultáneas.
			\end{itemize}
	\end{itemize}
	\begin{itemize}
		\item \textbf{Pizarra:}
			\begin{itemize}
				\item La pizarra soporta imagenes y archivos PowerPoint.
			\end{itemize}
	\end{itemize}
	\begin{itemize}
		\item \textbf{Calendario/Agenda:}
			\begin{itemize}
				\item Los profesores y los estudiantes pueden añadir eventos el el calendario del curso.
				\item El estudiante posee una página de inicio personal con los cursos en los cuales participa, todos los eventos o actividades de su calendario personal.
				\item El estudiante puede ver la calificación de las actividades realizadas, el total de puntos posibles, calificación del curso y comparar sus calificaciones con las del salón.
			\end{itemize}
	\end{itemize}
	\begin{itemize}
		\item \textbf{Capacidades de búsqueda:}
			\begin{itemize}
				\item El estudiante puede realizar búsquedas en todo el curso.
				\item El estudiante puede buscar en todos los hilos de discusión.
				\item Los estudiantes pueden restringir la búsqueda a través de filtros.
			\end{itemize}
	\end{itemize}
	\begin{itemize}
		\item \textbf{Trabajo desconectado:}
			\begin{itemize}
				\item Los estudiantes pueden ``compilar'' y descargar todo el contenido de un curso para ser impreso o guardado localmente.
			\end{itemize}
	\end{itemize}
	\begin{itemize}
		\item \textbf{Ayuda y Orientación del curso:}
			\begin{itemize}
				\item El sistema incluye tutoriales en línea para ayudar al estudiante a aprender el uso del sistema.
				\item Los estudiantes pueden acceder a un menú de ayuda para cualquiera de las herramientas del sistema.
			\end{itemize}
	\end{itemize}
	\begin{itemize}
		\item \textbf{Grupos de Trabajo:}
			\begin{itemize}
				\item El profesor puede asignar estudiantes a grupos.
				\item El sistema puede crear grupos, de forma aleatoria y con un cierto tamaño.
				\item Cada grupo puede tener su propio foro de discusión.
				\item Cada grupo puede tener chat o pizarra.
				\item A cada grupo se le pueden asignar tareas o actividades específicas.
				\item Los grupos pueden ser privados o monitoreados por el profesor.
				\item Los estudiantes escoger o seleccionar grupos.
			\end{itemize}
	\end{itemize}
	\begin{itemize}
		\item \textbf{Portafolios:}
			\begin{itemize}
				\item Los estudiantes pueden crear una página principal para cada curso.
			\end{itemize}
	\end{itemize}
	\begin{itemize}
		\item \textbf{Tipos de pruebas:}
			\begin{itemize}
				\item Tipos de pruebas: selección múltiple, respuestas múltiples, completar el espacio en blanco, matching, calculadas, encuestas, ensayos, ordenar una frase, preguntas que contengan imagenes, videos , audio,etc.
			\end{itemize}
	\end{itemize}
	\begin{itemize}	
		\item \textbf{Manejo automatizado de Pruebas:}
			\begin{itemize}
				\item El sistema soporta el editor MathML para la inclusión de fórmulas matemáticas en las preguntas.
				\item El acceso a las pruebas puede ser restringido por la dirección IP
				\item Los profesores pueden crear conjuntos de preguntas y organizarlas por dificultad o habilidad.
				\item Los profesores pueden importar y exportar preubas, quizzes, encuestas y autoevaluaciones para compartirlas con otros cursos.
				\item El sistema puede aleatorizar las preguntas  y respuestas.
				\item El profesor puede crear exámenes de prueba.
				\item El profesor puede definir tiempos límites para las evaluaciones.
				\item El profesor puede permitir múltiples intentos.
				\item Los estudiantes puede revisar intentos anteriores de una prueba.
				\item Los profesores definen si se debe mostrar las respuestas correctas.
			\end{itemize}
	\end{itemize}
	\begin{itemize}
		\item \textbf{Soporte para pruebas automatizadas:}
			\begin{itemize}
				\item Los profesores pueden crear un banco personal de pruebas.
				\item El sistema provee análisis de los datos de las pruebas.
				\item Se pueden importar preguntas desde bancos de pruebas que soporten QTI.
				\item Herramientas de puntuación online.
				\item Se puede puntuar cada estudiante en todas las preguntas o cada pregunta de todos los estudiantes.
			\end{itemize}
	\end{itemize}
	\begin{itemize}
		\item \textbf{Herramientas de puntuación online:}
			\begin{itemize}
				\item Los profesores pueden evaluar las respuestas de los estudiantes de forma anónima.
			\end{itemize}
	\end{itemize}
	\begin{itemize}
		\item \textbf{Control de calificaciones del curso (gradebooks):}
			\begin{itemize}
				\item Cuando el profesor agrega una actividad al curso, ésta se agrega automáticamente a la tabla de actividades a evaluar.
				\item Se puede crear una escala de evaluación que contemple porcentajes, estatus de aprobado/reprobado, calificación alfabética.
				\item El profesor puede agregar calificaciones para actividades off-line.
				\item Se pueden exportar las calificaciones contenidas en la tabla.
			\end{itemize}
	\end{itemize}
	\begin{itemize}
		\item \textbf{Manejo del Curso:}
			\begin{itemize}
				\item El instructor puede programar la aparición de eventos, tareas o pruebas, anuncios en el sistema de acuerdo a una fecha de inicio y una fecha de fin.
				\item Los profesores pueden personalizar el acceso a material específico del curso basado en el desempeño del estudiante.
				\item Los profesores pueden enlazar las discusiones a fechas específicas o eventos del curso.
			\end{itemize}
	\end{itemize}
	\begin{itemize}
		\item \textbf{Seguimiento del estudiante:}
			\begin{itemize}
				\item Historial de navegacion de cada estudiante.
				\item Los profesores pueden obtener reportes del número de veces, tiempo, fecha, frecuencia y dirección IP de cada estudiante que ha accesado al contenido de un curso, foro, tareas, etc.
				\item Los profesores pueden exportar toda la data de  seguimiento del estudiante así como compartirla con los mismos.
			\end{itemize}
	\end{itemize}
	\begin{itemize}
		\item \textbf{Compartir/Reusar contenido:}
			\begin{itemize}		
				\item Los estudiantes pueden exportar su página principal personal.
				\item Los profesores pueden compartir contenidos con otros profesores o estudiantes a través de un repositorio central de learning objects.
			\end{itemize}
	\end{itemize}
	\begin{itemize}
		\item \textbf{Plantillas del curso:}
			\begin{itemize}
				\item El sistema permite a los administradores usar un curso existente o una plantilla predefinida como base para un nuevo curso.
				\item El contenido del curso puede ser cargado a través de WebDAV.
				\item Personalización de la interfaz:
				\item El sistema proporciona plantillas (por defecto) para la interfaz de los cursos.
				\item Los profesores pueden modificar los iconos y colores de un curso.
				\item El profesor puede cambiar el orden y los nombres de los items ubicados en el menú del curso.
				\item El sistema puede soportar múltiples instituciones, departamentos, escuelas u otras organizaciones en una sola instalación donde cada unidad puede tener su estilo y sus plantillas así como imágenes, encabezados u pie de páginas.
			\end{itemize}
	\end{itemize}
	\begin{itemize}
		\item \textbf{Herramientas de diseño de cursos:}
			\begin{itemize}
				\item Los profesores pueden organizar los learning objects, las herramientas del curso y el contenido del material que puede ser reusable.
				\item Reutilización de los cursos como plantilla para otros cursos en los cuales se maneje la misma información.
			\end{itemize}
	\end{itemize}
	\begin{itemize}
		\item \textbf{Requerimientos de base de datos:}
			\begin{itemize}
				\item El sistema soporta Oracle.
				\item El sistema soporta MS SQL Server
			\end{itemize}
	\end{itemize}
	\begin{itemize}
		\item \textbf{Comunidad de Usuarios:}\\ 
		\citep{BB_COM2008}\\
		Algunas universidades que usan esta plataforma son: 
		\begin{itemize}
			\item Universidad Baylor de Medicina, Texas
			\item Universidad de Carolina del Este
			\item Colegio Mount Sinai de Medicina, New York
			\item Universidad de Ciencias de la Salud Oregon, Oregon
			\item Escuela Superior de Medicina Osteopática de Filadelfia, Filadelfia
			\item Universidad Central del Caribe, Puerto Rico
			\item Universidad de Chicago-Escuela de Enfermeras, Illinois
			\item Universidad de Colorado Centrado en ciencias de la Salud, Colorado
			\item Universidad de Biología de Maryland Chesapeake, Maryland
			\item Universidad de Tennessee-Memphis,Tennessee
			\item Universidad de Texas Houston especializada en Odontología, Texas entre otras Universidades.
		\end{itemize}
	\end{itemize}
	
\subsection{Desire2Learn 8.2}
	\paragraph{Enfoque Pedagógico:} Cognitivismo, constructivismo.
	\paragraph{Características:}
	\begin{itemize}
		\item \textbf{Foros:}
			\begin{itemize}
				\item Herramienta de revisión ortográfica.
				\item Asociar una discusión al contenido de un curso
				\item Los posts pueden incluir archivos media, ecuaciones, archivos adjuntos o direcciones URL.
				
			\end{itemize}
	\end{itemize}
	\begin{itemize}
		\item \textbf{Manejo de discusiones:}
			\begin{itemize}
				\item Los profesores pueden permitir a los estudiantes la creción de grupos de discusión.
				\item Las discusiones pueden ser compartidas entre cursos, departamentos y cualquier otra unidad de la institución.
			\end{itemize}
	\end{itemize}
	\begin{itemize}
		\item \textbf{Intercambio de Archivos:}
			\begin{itemize}
				\item Entrega de tareas a través de ``drop boxes'' o casilleros del profesor.
				\item Los administradores pueden definir el tamaño de espacio en disco para cada usuario.
				\item Los estudiantes tienen una carpeta privada en la cual pueden almacenar sus archivos.
				\item Detección de virus en el proceso de upload/download de archivos.
				\item Posibilidad de compartir archivos.
			\end{itemize}
	\end{itemize}
	\begin{itemize}
		\item \textbf{Correo Interno:}
			\begin{itemize}
				\item Envío de mensajes a toda la clase.
				\item Libreta de direcciones.
				\item Capacidad de hacer forward a correo extrerno.
				\item Los estudiantes pueden colocar notas a cualquier página.
				\item Posibilidad de que los estudiantes agreguen al material del curso sus apuntes para crear una guía de estudio.
			\end{itemize}
	\end{itemize}
	\begin{itemize}
		\item \textbf{Chat:}
			\begin{itemize}
				\item Soporte ilimitado de grupos de discusión simultáneos.
				\item El sistema crea archivos de registro para todas las salas de chat.
				\item Los estudiantes pueden crear nuevas salas de chat.
				\item Los profesores pueden moderar el chat y suspender estudiantes de las salas de chat.
				\item Soporta un número limitado de salas simultáneas.
			\end{itemize}
	\end{itemize}
	\begin{itemize}
		\item \textbf{Pizarra:}
			\begin{itemize}
				\item La pizarra soporta imágenes y archivos PowerPoint.
			\end{itemize}
	\end{itemize}
	\begin{itemize}
		\item \textbf{Calendario/Agenda:}
			\begin{itemize}
				\item Los profesores y los estudiantes pueden añadir eventos el el calendario del curso.
				\item El estudiante posee una página de inicio personal con los cursos en los cuales participa, todos los eventos o 		actividades de su calendario personal.
				\item El estudiante puede ver la calificación de las actividades realizadas, el total de puntos posibles, calificación del curso y comparar sus calificaciones con las del salón.
			\end{itemize}
	\end{itemize}
	\begin{itemize}
		\item \textbf{Capacidades de búsqueda:}
			\begin{itemize}
				\item El estudiante puede realizar búsquedas en todo el curso.
				\item El estudiante puede buscar en todos los hilos de discusión.
			\end{itemize}
	\end{itemize}
	\begin{itemize}
		\item \textbf{Trabajo desconectado:}
			\begin{itemize}
				\item Los estudiantes pueden ``compilar'' y descargar todo el contenido de un curso para ser impreso o guardado localmente.
			\end{itemize}
	\end{itemize}
	\begin{itemize}
		\item \textbf{Ayuda y Orientación del curso:}
			\begin{itemize}
				\item El sistema incluye tutoriales en línea para ayudar al estudiante a aprender el uso del sistema.
				\item Los estudiantes pueden acceder a un menú de ayuda para cualquiera de las herramientas del sistema.
			\end{itemize}
	\end{itemize}
	\begin{itemize}
		\item \textbf{Grupos de Trabajo:}
			\begin{itemize}
				\item El profesor puede asignar estudiantes a grupos.
				\item El sistema puede crear grupos, de forma aleatoria y con un cierto tamaño.
				\item Cada grupo puede tener su propio foro de discusión.
				\item Cada grupo puede tener chat o pizarra.
				\item A cada grupo se le pueden asignar tareas o actividades específicas.
				\item Los grupos pueden ser privados o monitoreados por el profesor.
				\item Los estudiantes escoger o seleccionar grupos.
				\item Cada grupo puede tener su carpeta de presentaciones, lista de emails, encuestas, actividades, compartir calendario de eventos, intercambiar archivos y designar un líder de grupo.
			\end{itemize}
	\end{itemize}
	\begin{itemize}
		\item \textbf{Comunidad:}
			\begin{itemize}
				\item Los estudiantes, de distintos cursos, pueden participar en todas las salas de chat o foros del sistema.
			\end{itemize}
	\end{itemize}
	\begin{itemize}
		\item \textbf{Portafolios:}
			\begin{itemize}
				\item Los estudiantes pueden crear una página principal para cada curso.
			\end{itemize}
	\end{itemize}
	\begin{itemize}
		\item \textbf{Tipos de pruebas:}
			\begin{itemize}
				\item Tipos de pruebas: selección múltiple, respuestas múltiples, completar el espacio en blanco, matching, ordenamientos, calculadas, encuestas, ensayos, preguntas que contengan imágenes, videos , audio,etc. Se pueden personalizar el tipo de preguntas a realizar.
				
			\end{itemize}
	\end{itemize}
	\begin{itemize}
		\item \textbf{Manejo automatizado de pruebas:}
			\begin{itemize}
				\item El sistema puede aleatorizar las preguntas  y respuestas.
				\item El profesor puede crear exámenes de prueba.
				\item El profesor puede definir tiempos límites para las evaluaciones.
				\item El profesor puede permitir múltiples intentos.
				\item Los estudiantes puede revisar intentos anteriores de una prueba.
				\item Los profesores definen si se debe mostrar las respuestas correctas.
				\item El sistema soporta la presentación de ecuaciones matemáticas con LaTeX o MathML.
				\item El acceso a las pruebas puede ser restringido por dirección IP.
				\item Los profesores pueden hacer conjuntos de preguntas, organizándolos por difcultad y  habilidad.
				\item Se puede importar y exportar pruebas, encuestas y auto evaluaciones para compartirlas con otros cursos.
			\end{itemize}
	\end{itemize}
	\begin{itemize}
		\item \textbf{Soporte para pruebas automatizadas:}
			\begin{itemize}
				\item Los profesores pueden crear un banco personal de pruebas.
				\item El sistema provee análisis de los datos de las pruebas.
				\item Se pueden importar preguntas desde bancos de pruebas que soporten QTI.
			\end{itemize}
	\end{itemize}
	\begin{itemize}
		\item \textbf{Herramientas de puntuación online:}
			\begin{itemize}
				\item \textbf{Instructors can enable students to rate and comment on submissions of other students.}
			\end{itemize}
	\end{itemize}
	\begin{itemize}
		\item \textbf{Control de calificaciones del curso (gradebooks):}
			\begin{itemize}
				\item Se puede crear una escala de evaluación que contemple porcentajes, estatus de aprobado/reprobado, calificación alfabética.
				\item El profesor puede agregar calificaciones para actividades off-line.
				\item Se pueden exportar las calificaciones contenidas en la tabla.
				\item El software, automaticamente, calcula la nota mínima, máxima y promedio en cualquier item a evaluar incluyendo tareas y quices.
			\end{itemize}
	\end{itemize}
	\begin{itemize}
		\item \textbf{Manejo del Curso:}
			\begin{itemize}
				\item El instructor puede programar la aparición de eventos, tareas o pruebas, anuncios en el sistema de acuerdo a una fecha de inicio y una fecha de fin.
				\item Los profesores pueden enlazar las discusiones a fechas específicas o eventos del curso.
				\item Los profesores pueden personalizar el acceso a material específico del curso basado en el desempeño del estudiante.
			\end{itemize}
	\end{itemize}
	\begin{itemize}
		\item \textbf{Seguimiento del estudiante:}
			\begin{itemize}
				\item Historial de navegacion de cada estudiante.
				\item Los profesores pueden obtener reportes del número de veces, tiempo, fecha, frecuencia y dirección IP de cada estudiante que ha accesado al contenido de un curso, foro, tareas, etc.
			\end{itemize}
	\end{itemize}
	\begin{itemize}
		\item \textbf{Compartir/Reusar contenido:}
			\begin{itemize}
				\item Los profesores pueden compartir contenidos con otros profesores o estudiantes a través de un repositorio central de learning objects.
			\end{itemize}
	\end{itemize}
	\begin{itemize}
		\item \textbf{Plantillas del curso:}
			\begin{itemize}
				\item El sistema permite a los administradores usar un curso existente o una plantilla predefinida como base para un nuevo curso.
				\item El contenido del curso puede ser cargado a través de WebDAV.
			\end{itemize}
	\end{itemize}
	\begin{itemize}
		\item \textbf{Personalización de la interfaz:}
			\begin{itemize}
				\item El sistema proporciona plantillas (por defecto) para la interfaz de los cursos.
				\item Los profesores pueden modificar los iconos y colores de un curso.
				\item El profesor puede cambiar el orden y los nombres de los items ubicados en el menú del curso.
				\item El sistema puede soportar múltiples instituciones, departamentos, escuelas u otras organizaciones en una sola instalación donde cada unidad puede tener su estilo y sus plantillas así como imágenes, encabezados u pie de páginas.
				\item Los usuarios pueden seleccionar cualquier página como página principal de Contenidos.
			\end{itemize}
	\end{itemize}
	\begin{itemize}
		\item \textbf{Herramientas de diseño de cursos:}
			\begin{itemize}
				\item Los profesores pueden organizar los learning objects, las herramientas del curso y el contenido del material que puede ser reusable.
				\item Reutilización de los cursos como plantilla para otros cursos en los cuales se maneje la misma información.
			\end{itemize}
	\end{itemize}
	\begin{itemize}
		\item \textbf{Requerimientos de base de datos:}
			\begin{itemize}
				\item El sistema soporta MS SQL Server
			\end{itemize}
	\end{itemize}
	
\section{Resumen}
Entre los LMS estudiados se observan algunas herramientas con características comunes, entre ellas se encuentran:
\begin{itemize}
			\item \textbf{Foro:} herramienta de revisión ortográfica, posibilidad de habilitar/deshabilitar el recibir los posts como e-mails.
			\item \textbf{Correo Interno:} envío de mensajes a toda la clase, capacidad de hacer forward a correo extrerno.
			\item \textbf{Chat:} el sistema crea archivos de registro para todas las salas de chat.
			\item \textbf{Manejo de discusiones:} los profesores pueden permitir a los estudiantes la creción de grupos de discusión.
			\item \textbf{Calendario/Agenda:} 
				\begin{itemize}
					\item el estudiante posee una página de inicio personal con los cursos en los cuales participa, todos los eventos o actividades de su calendario personal. 
					\item El estudiante puede ver la calificación de las actividades realizadas, el total de puntos posibles, calificación del curso y comparar sus calificaciones con las del salón.
				\end{itemize}
			\item \textbf{Intercambio de Archivos:} Entrega de tareas a través de ``drop boxes'' o casilleros del profesor.
			\item \textbf{Capacidades de búsqueda:} El estudiante puede buscar en todos los hilos de discusión.
			\item \textbf{Ayuda y Orientación del curso:} Los estudiantes pueden acceder a un menú de ayuda para cualquiera de las herramientas del sistema.
			\item \textbf{Comunidad:} Los estudiantes, de distintos cursos, pueden participar en todas las salas de chat o foros del sistema.
\end{itemize}