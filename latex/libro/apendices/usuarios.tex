%!TEX root = ../Libro.tex
\chapter{Stakeholders y Usuarios}
\label{Usuarios}
\section{Stakeholders}
En la tabla que se muestra a continuación, se ofrece un breve resumen de los stakeholders que participan en el desarrollo de este proyecto.\\

\begin{longtable}{p{0.2\textwidth}p{0.35\textwidth}p{0.35\textwidth}}
	\multicolumn{1}{c}{\textbf{Nombre}} &
	\multicolumn{1}{c}{\textbf{Descripción}} &
	\multicolumn{1}{c}{\textbf{Responsabilidades}} \\ \hline
\endhead \endfoot
	DSM
	&
	Moderador técnico. Es la unidad encargada de
	dar apoyo, servicio y asesoría en el uso de recursos
	multimedia para programas de docencia,
	investigación y extensión de la USB
	&
	\vspace{-0.7cm}
		\begin{itemize}
			\item Garantizar el mantenimiento del sistema
			\item Hacer seguimiento al desarrollo del sistema
			\item Garantizar la disponibilidad de los recursos necesarios para el funcionamiento del sistema
		\end{itemize}
	\\ \hline
	
	\raggedright
	Fidel Gil \linebreak Hermes Rodríguez
	&
	Mentor y Asesor, respectivamente.
	Prestan asesoría en cuanto a la determinación de los requerimientos
	del sistema y de las características de su comportamiento.
	&
	\vspace{-0.7cm}
	\begin{itemize}
		\item Prestar asesoría en la determinación de los requerimientos y características del sistema.
		\item Monitoreo del progreso del sistema
	\end{itemize}
	\\ \hline
	
	\raggedright
	Ana Gabriela Díaz \linebreak José Lorenzo Rodríguez \linebreak Joaquín Windmüller
	&
	Desarrollador. Encargado del diseño e implementación del Sistema
	&
	\vspace{-0.7cm}
	\begin{itemize}
		\item Desarrollo de la aplicación.
		\item Aporte de ideas para mejorar la funcionalidad de la aplicación.
	\end{itemize}	
	\\ \hline
\caption{Resumen de los Stakeholders} \label{tab:stakeholders} \\  
\end{longtable}

\section{Usuarios}
En la tabla que se muestra a continuación, se ofrece un breve resumen de los usuarios que maneja el sistema.
\begin{longtable}{p{0.2\textwidth}p{0.35\textwidth}p{0.35\textwidth}}
	\multicolumn{1}{c}{\textbf{Nombre}} &
	\multicolumn{1}{c}{\textbf{Descripción}} &
	\multicolumn{1}{c}{\textbf{Responsabilidades}} \\ \hline
\endhead \endfoot

	Administrador
	&
	Usuario del sistema que posee todos los permisos
	&
	\vspace{-0.7cm}
	\begin{itemize}
		\item Realizar el mantenimiento del sistema
		\item Incorporar nuevas funcionalidades al sistema, de ser necesario.
		\item Mejorar las herramientas disponibles.
	\end{itemize}
	\\ \hline	
	Profesor
	&
	Persona con habilidades pedagógicas encargada de impartir
	la educación. Usuario principal o clave del sistema que 
	tiene la posibilidad de usarlo como herramienta para impartir un curso
	&
	\begin{itemize}
		\item Crear, actualizar, modificar o gestionar la información pedagógica que se quiera utilizar como ayuda de la metodología seleccionada para impartir el curso.
		\item Hacer seguimiento de las actividades planificadas
		\item Evaluar el desempeño del estudiante.
	\end{itemize}
	\\ \hline	
	Facilitador
	&
	Ayudante del profesor que contribuye con el proceso de enseñanza
	&
	\begin{itemize}
		\item Se le atribuyen responsabilidades similares a las del profesor.
	\end{itemize}
	\\ \hline	
	Estudiante
	&
	Persona que cursa estudios
	&
	\begin{itemize}
		\item Contribuir en el desarrollo de las actividades del curso.
		\item Hacer uso de las herramientas del sistema propuestas para complementar el proceso de aprendizaje.
		\item Evaluar o dar realimentación sobre el sistema y sus características.
	\end{itemize}
	\\ \hline
	\caption{Resumen de los Usuarios} \label{tab:usuarios} \\  
\end{longtable}