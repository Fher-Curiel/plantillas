%!TEX root = ../Libro.tex
\chapter{Información Complementaria}
\section{Learning Objects}
Es un término amplio que se refiere a:

\begin{itemize}
	\item ``Cualquier entidad, digital o no, que puede ser usado para el aprendizaje, educación o entrenamiento.''
	\item ``Cualquier recurso que puede ser reusado para dar soporte al aprendizaje.''
	\item ``Una entidad digital que puede ser usada, reusada o referenciada durante el aprendizaje apoyado por la tecnología.'' 
\end{itemize}

Estas entidades educativas son descritas por un modelo de datos estandarizado, codificado con XML. El propósito de este modelo de datos es facilitar el descubrimiento de los contenidos, usualmente dentro del contexto de un LMS.\\

Las principales características de un Learning Object son:
\begin{itemize}
	\item \textbf{Auto-contenido y reusable}\\
	Cada learning object es independiente y puede ser compartido y reusado, en distintas ocasiones y para distintos fines, como una entidad indivisible.
	\item \textbf{Entidad de aprendizaje reducida}\\
	Al contrario de una clase completa, los learning object contienen recursos más breves de aprendizaje de unos 2 a 15 minutos.
	\item \textbf{Pueden ser agregados}\\
	Los learning objects pueden ser agrupados en colecciones más grandes, incluso en estructuras tradicionales de educación.
	\item \textbf{Rastreados con metadatos}\\
	Cada learning object contiene información descriptiva sobre la que se pueden realizar búsquedas.
\end{itemize}

Los principales usos de este estándar son para permitir...
\begin{itemize}
	\item A a los estudiantes e instructores, buscar, evaluar, adquirir y utilizar objetos de aprendizaje.
	\item El intercambio de Learning Objects entre LMS que soporten la tecnología.
	\item A organizaciones educativas, expresar los contenidos educativos en una manera estandarizada e independiente del contenido en sí.
	\item La creación de objetos de aprendizaje en unidades que puedan ser combinadas en formas significativas.
\end{itemize}
\citep{IEEE_LOM}