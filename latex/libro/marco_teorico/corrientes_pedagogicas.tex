%!TEX root = ../Libro.tex
\section{Corrientes Pedagógicas}
Las corrientes pedagógicas son distintas teorías de pensamiento, surgidas a lo largo de la historia, que intentan explicar el proceso de aprendizaje y formación del ser humano, las más importantes son:
\begin{itemize}
	\item Conductismo
	\item Cognitivismo
	\item Constructivismo
	\item Constructivismo social
	\item Conectivismo
\end{itemize}

\subsection{Conductismo}
Según \citeauthor{Conductismo_Manosalva1996}, en el conductismo la experiencia precede al conocimiento, en otras palabras, define el aprendizaje como un cambio observable en el comportamiento producto de una relación ``estímulo - respuesta''. Los procesos internos tales como el pensamiento y la motivación son incognoscibles \citep{Conductismo_Manosalva1996}, no pueden ser observados ni medidos directamente por lo que no son relevantes a la investigación científica del aprendizaje.
El aprendizaje se constata por medio de la observación de un cambio en el comportamiento. Si no hay cambio observable no hay aprendizaje.\\

El conductismo aprueba el uso de refuerzos para fortalecer conductas apropiadas y su desuso para debilitar las no deseadas. La asignación de calificaciones, recompensas y castigos son también aportaciones de esta teoría.\\

Los principios de las ideas conductistas pueden aplicarse con éxito en la adquisición de conocimientos memorísticos que suponen niveles primarios de comprensión, como por ejemplo el aprendizaje de las capitales del mundo o las tablas de multiplicar. Sin embargo esto presenta una limitación importante: la repetición no garantiza asimilación de la nueva conducta, sino sólo su ejecución. Por ejemplo, una persona puede aprender las tablas de multiplicar pero no sabrá cuando debe hacer uso de ellas. Esto indica que la situación aprendida no se puede extrapolar fácilmente a otras situaciones.\\

En este paradigma, el estudiante es un individuo cuyo aprendizaje se puede modificar por medio de un reajuste de los recursos educativos de tal manera que éste adquiera las actitudes académicas deseadas.\\

El maestro es el encargado de ejecutar dichos reajustes así como aplicar los estímulos necesarios para la enseñanza.

\subsection{Cognitivismo}
El paradigma cognitivista sustenta al aprendizaje como un proceso en el cual se sucede la modificación de significados de manera interna, producido intencionalmente por el individuo como resultado de la interacción entre la información procedente del medio y el sujeto activo. Dicha perspectiva surge a finales de 1960 como una transición entre el paradigma conductista y las actuales teorías psicopedagógicas.\\

``Al cognitivismo le interesa la representación mental y por ello las categorías o dimensiones de lo cognitivo: la atención, la percepción, la memoria, la inteligencia, el lenguaje, el pensamiento y para explicarlo puede, y de hecho acude a múltiples enfoques, uno de ellos es el de procesamiento de la información; y cómo las representaciones mentales guían los actos (internos o externos) del sujeto con el medio, pero también cómo se generan (construyen) dichas representaciones en el sujeto que conoce.'' \citep{Cognitivismo_Ferreiro1996}\\

El cognitivismo es, de manera simplificada, el proceso independiente de decodificación de significados que conduzcan a la adquisición de conocimientos a largo plazo y al desarrollo de estrategias que permitan la libertad de pensamiento, la investigación y el aprendizaje continuo en cada individuo, lo cual da un valor real a cualquier cosa que se desee aprender. De aquí entonces se desprende el paradigma del Cognitivismo, ``un marco global de referencia para el crecimiento y desarrollo personal'' \citep{Cognitivismo_Ferreiro1996}

\subsection{Constructivismo}
Esta corriente pedagógica afirma que el conocimiento de todas las cosas es un proceso mental del individuo, que se desarrolla de manera interna conforme obtiene información y se relaciona con su entorno.\\

Para el constructivismo, el aprendizaje es un proceso en el cual el estudiante construye activamente nuevas ideas o conceptos basados en conocimientos presentes y pasados. En otras palabras, ``el aprendizaje se forma construyendo nuestros propios conocimientos desde nuestras propias experiencias'' \citep{Constructivismo_Ormrod2003}\\

A pesar de ser el aprendiz quien construye su aprendizaje, el profesor juega un rol importante en el proceso, actuando como facilitador y animando a los estudiantes a solucionar problemas reales o simulaciones. Por lo general el aprendizaje se hace por medio de un proceso social de colaboración entre los estudiantes y bajo esta óptica, el proceso de aprendizaje no es de ``todo o nada'' sino que los estudiantes desarrollan nuevos conocimientos a partir de los que ya poseen. Esto significa que el docente debe constatar que el aprendizaje efectivo logrado por el aprendiz es el esperado así como fomentar la conexión entre los estudiantes y plantear situaciones para estimular el razonamiento.\\

El constructivismo propone una metodología de aprendizaje, más no propone los medios para llevar a cabo dicho aprendizaje. Es decir, describe cómo ocurre pero no cómo se logra el aprendizaje.

\subsection{Constructivismo social}
El constructivismo social en educación y teoría del aprendizaje estudia la forma en que el ser humano aprende a la luz de su situación social y de la comunidad en la que aprende.\\

Uno de los puntos focales del constructivismo es descubrir las maneras en las que los individuos y grupos participan en la creación de la percepción de su realidad social. El estudio de la aparición de fenómenos sociales, su institucionalización y como posteriormente se tornan en tradiciones. La realidad construida socialmente, por los individuos y su interpretación de sus conocimientos, es un proceso dinámico y constante. \\

En décadas recientes, los teóricos del constructivismo han extendido en enfoque tradicional del aprendizaje para referirse a las dimensiones colaborativas y sociales del aprendizaje.  \\

\citeauthor{Holmes2001}, propone extender el constructivismo social de modo que se tome en cuenta la sinergía entre los más recientes avances de la tecnología de la información. Introduce el término Constructivismo Comunal, en el cual ``los estudiantes no solamente pasan a través de un curso, como el agua a través de una tubería; sino que dejan su propia huella en el proceso de aprendizaje, en su escuela o universidad e idealmente en su disciplina. Esto resultará en un beneficio para el curso o la institución, pero más importante es el beneficio que genera a los estudiantes.'' Avances como los blogs, wikis y podcasts, aumentan el potencial comunicacional de los individuos, lo cual permite que el conocimiento construido individualmente redunde en un beneficio a los demás aprendices.\citep{Holmes2001} \\

El constructivismo social expone que el ambiente de aprendizaje óptimo es aquel donde una interacción dinámica entre los instructores y los alumnos, con actividades que proveen oportunidades para los alumnos de crear su propia verdad gracias a la interacción con lo demás. Esta teoría, por lo tanto, enfatiza la importancia de la cultura y el contexto para el entendimiento de lo que está sucediendo en la sociedad y para construir conocimiento basado en este entendimiento \citep{Carretero1997}. \\

Según \citeauthor{Glasersfeld1989}, el constructivismo social plantea dos principios fundamentales cuya aplicación tiene consecuencias tanto en el estudio del desarrollo cognitivo como en la práctica educativa:

\begin{itemize}
	\item El conocimiento no se recibe pasivamente sino que es construido activamente por el sujeto cognitivo.
	\item La función del aprendizaje es adaptable. Sirve para la organización del mundo de la experiencia y no para el descubrimiento de una realidad ontológica \citep{Glasersfeld1989}
\end{itemize}

\subsection{Conectivismo}
El conectivismo es una teoría del aprendizaje para la era digital que ha sido desarrollada por George Siemens basado en el análisis de las limitaciones del conductismo, el cognitivismo y el constructivismo, para explicar el efecto que la tecnología ha tenido sobre la manera en que actualmente vivimos, nos comunicamos y aprendemos. \citep{Conectivismo_Siemens2004}\\

El conectivismo es la integración de los principios explorados por la teorías del caos, redes neuronales, complejidad y auto-organización. El aprendizaje, según \citeauthor{Conectivismo_Siemens2004}, es un proceso que ocurre dentro de una amplia gama de ambientes que no no están necesariamente bajo el control del individuo, es por esto que el conocimiento puede residir fuera del ser humano, por ejemplo dentro de una organización o una base de datos, y se enfoca en la conexión especializada en conjuntos de información que nos permita aumentar cada vez más nuestro estado actual de conocimiento.\\

Esta teoría es conducida por el entendimiento de que las decisiones están basadas en las transformación acelerada de los basamentos. Continuamente nueva información es adquirida dejando obsoleta la anterior. En este contexto, es vital adquirir la capacidad para discernir entre lo importante y lo trivial, así como la capacidad para reconocer cuando esta nueva información altera las decisiones tomadas en base a la información pasada.\\

El punto de inicio del conectivismo es el individuo. El conocimiento personal se hace de una red, que alimenta de información a organizaciones e instituciones, que a su vez agregan nueva información en la misma red, lo cual termina proveyendo nuevo aprendizaje al individuo. Este ciclo de desarrollo del conocimiento permite a los aprendices a mantenerse actualizados en el campo en el cual han formado conexiones.

\paragraph{Principios del conectivismo}
\begin{itemize}
	\item El aprendizaje y el conocimiento yacen en la diversidad de opiniones.
	\item El aprendizaje es el proceso de conectar nodos o fuentes de información.
	\item No solo los humanos aprenden y el conocimiento puede residir fuera del ser humano.
	\item La capacidad de aumentar el conocimiento es más importante que la información conocida.
	\item Nutrir y mantener las conexiones es necesario para facilitar el aprendizaje continuo.
	\item La habilidad para ver las conexiones entre los campos, ideas y conceptos es primordial.
	\item La información actualizada y precisa es la intención de todas las actividades del proceso conectivista.
	\item La toma de decisiones es en sí misma un proceso de aprendizaje. Escoger qué aprender y el significado de la información entrante es visto a través del lente de una realidad cambiante: es posible que lo que hoy es correcto mañana esté errado bajo la nueva información que se recibe.
\end{itemize}

Con respecto al primer punto, \citeauthor{Conectivismo_Siemens2005} indica que una red de aprendizaje puede sostener puntos aparentemente contradictorios dado que lo que hoy es cierto mañana podría no serlo ante algún cambio en la base. Y es esta diversidad la aumenta la posibilidad de realizar decisiones acertadas. \citep{Conectivismo_Siemens2005}