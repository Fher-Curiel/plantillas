%!TEX root = ../Libro.tex
\section{Accesibilidad}
La accesibilidad se refiere a la facilidad con la cual algo puede ser usado o accedido por cualquier personas, especialmente por aquellas que poseen algún tipo de discapacidad.\\

La accesibilidad está ligada al diseño universal, que es un concepto relativamente nuevo donde se busca que el proceso de diseño se realice teniendo en cuenta a personas con distintos niveles de habilidad. El diseño universal propone 7 principios: \citep{Accesibilidad_DU1997}
\begin{enumerate}
	\item Uso equitativo
	\item Flexibilidad en el uso
	\item Uso simple e intuitivo
	\item Información perceptible
	\item Tolerancia a errores
	\item Esfuerzo físico bajo
	\item Tamaño y espacio apropiado para el uso
\end{enumerate}

\subsection{Accesibilidad a los contenidos en la Web}
``La accesibilidad Web significa que personas con algún tipo de discapacidad van a poder hacer uso de la Web. En concreto, al hablar de accesibilidad Web se está haciendo referencia a un diseño Web que va a permitir que estas personas puedan percibir, entender, navegar e interactuar con la Web, aportando a su vez contenidos. La accesibilidad Web también beneficia a otras personas, incluyendo personas de edad avanzada que han visto mermadas sus habilidades a consecuencia de la edad'' \citep{Accesibilidad_IntroW3C2005} \\

La labor de desarrollar aplicaciones web accesibles es ardua debido a la naturaleza del ámbito: cualquier persona podría acceder al sistema utilizando, potencialmente, infinidad de dispositivos distintos como navegadores, lectores de pantalla, etc. Para lograr estandarizar el proceso de desarrollo web se creó el \emph{Consorcio World Wide Web} (W3C por sus siglas en inglés), el cual es el encargado de desarrollar dichos estándares \citep{Accesibilidad_DefW3C2008}. Entre dichos estándares Web están las directrices para asegurar y mejorar la accesibilidad a los contenidos por parte de los usuarios.\\

Facilitar el acceso a los contenidos en la Web es importante ya que esta una fuente de información que se usa en muchos contextos como la educación, el comercio, gobierno y más \citep{Accesibilidad_IntroW3C2005}.\\

En el desarrollo web, las pautas para determinar el nivel de accesibilidad de un sitio son definidas bajo 3 niveles de prioridad:
\begin{enumerate}
 \item \textbf{Prioridad 1} reglas que \textbf{deben} ser cumplidas por los desarrolladores web.
 \item \textbf{Prioridad 2} reglas que \textbf{deberían} ser cumplidas por los desarrolladores web.
 \item \textbf{Prioridad 3} reglas que \textbf{pueden} ser cumplidas por los desarrolladores web.
\end{enumerate}

Cumplir las reglas de cada uno de estos niveles asegura al desarrollador que las barreras de acceso serán cada vez menores. Las pautas a seguir están disponibles en el apéndice \ref{apendice_accesibilidad}