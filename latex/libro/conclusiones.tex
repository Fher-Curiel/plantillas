%!TEX root = Libro.tex
\chapter{Conclusiones y recomendaciones}
	El sistema de Gestión del Aprendizaje Ósmosis2 fue diseñado e implementado considerando dos aspectos muy importantes. A nivel técnico se entrega un software fácil de mantener con una separación clara y organizada entre sus capas; y a nivel pedagógico presenta un concepto innovador en el área de \emph{e-learning} a través de la integración del conectivismo y los principios de la Web 2.0 con los modelos clásicos de aprendizaje.\\
	
	Para ello fue necesario realizar una investigación sobre las teorías del aprendizaje vigentes y bajo las cuales se han desarrollado otros LMS que están actualmente en el mercado. Así mismo, y con el propósito de seleccionar aquellas herramientas que se adaptaran a los conceptos bajo los cuáles se construiría Ósmosis2, se realizó un análisis comparativo de los diferentes LMS, incluyendo la versión 1.5 de Ósmosis, destacando aquellas caraterísticas comunes y que son esenciales en todo sistema de aprendizaje.\\

	En función de las características antes señaladas se seleccionaron aquellas herramientas que se consideraban necesarias para la nueva implementación y se trató de vincularlas con enfoque pedagógicos específicos, que en conjunto, permitieran crear una nueva visión del sistema. De la misma manera se realizó un estudio sobre los principios de la Web 2.0 y de cómo podían adaptarse a la nueva filosofía que se quería incorporar. \\
	
	Para reforzar la concepción sobre la cuál se levantó el sistema (una simbiosis de los modelos clásicos de aprendizaje con el conectivismo) se decidió que cada una de las herramientas sería un plugin separado, dándole libertad al profesor para seleccionar aquellas que se adapten a sus objetivos. \\

	Por otra parte, como resultado de la investigación realizada durante el presente proyecto de grado y con el objetivo de complementar las funcionalidades implementadas, se proponen las siguientes recomendaciones y mejoras al sistema:

\section*{Nuevas funcionalidades}
\begin{itemize}
	\item Complementar el diseño actual de los cursos con plantillas elaboradas en función de los distintos enfoques pedagógicos presentados. Estas plantillas utilizarían como herramientas principales aquellas que se puedan asociar directamente con el enfoque que representan, además podrían ser  ``personalizadas'' incorporando otros recursos. De esta manera los profesores tienen la posibilidad de escoger el modelo de curso que se adapte mejor a sus necesidades y al enfoque con el cuál se identifican.
	\item Incorporar a la aplicación la capacidad de crear, modificar o eliminar grupos de trabajo, así como como extender las herramientas implementadas para aceptar este nuevo nivel de organización.
	\item Implementar un herramienta que le permita al profesor mantener un acta de calificaciones en la cual pueda llevar un control del desempeño del estudiante en las actividades del curso.
	\item Suministrar a los usuarios una libreta de contactos que permita mantener los datos de otros usuarios conocidos.
	\item Habilitar la opción de exportar los contenidos de las herramientas para poder ser consultados de sin conexión.
	\item Desarrollar recursos como la Pizarra, que permitan hacer uso de medios audiovisuales para llevar a cabo, de forma interactiva, la enseñanza a distancia. 
	\item Agregar una herramienta que permita visualizar las estadísticas del uso del sistema: por curso, por herramienta y seguimiento de usuarios.
	\item Permitir a cada usuario mantener un portafolio o currículo de las actividades académicas, laborales y extra-académicas en las que ha participado.
	\item Permitir a Ósmosis exponer sus recursos en forma de Learning Objects.
	\item Crear una herramienta que le permita a los usuarios registrar enlaces o fuentes de noticias (RSS) con material relacionado a un curso.
	\item Dar soporte, de manera global, para el almacenamiento de borradores al redactar mensajes de modo que los usuarios tengan al posibilidad de reanudar su trabajo posteriormente.
	\item Permitir a los dueños del curso exportar e importar los contenidos del curso.
\end{itemize}

\section*{Extensión a las herramientas existentes}
\subsection*{Blog}
\begin{itemize}
	\item Mejorar la interconexión de los blogs por medio de Trackback y Pingback.
	\item Permitir a los usuarios importar o sincronizar su blog en Ósmosis con uno externo.
	\item Agregar la posibilidad de publicaciones multimedia: Screencasts y Podcast.
\end{itemize}

\subsection*{Pruebas}
\begin{itemize}
	\item Permitir al profesor preparar y programar la aplicación de pruebas.
	\item Completar la implementación del estándar \emph{QTI} para la definición de pruebas en línea. Esto permitiría a Ósmosis interoperar con otras aplicaciones y LMS existentes.
\end{itemize}

\subsection*{Chat}
\begin{itemize}
	\item Agregar la posibilidad de registrar, de manera automatizada, las preguntas interesantes que se generen en el chat.
	\item Implementar la sala de chat para el curso completo con la posibilidad de expulsión de un usuario y el control de quienes pueden escribir.
\end{itemize}

\subsection*{Foro}
\begin{itemize}
	\item Incorporar la capacidad de calificar los mensajes de modo que se destaquen las respuestas inteligentes.
	\item Permitir al profesor programar la fecha de cierre de las discusiones.
\end{itemize}

\subsection*{Casillero}
\begin{itemize}
	\item Incluir de manera nativa el control de versiones sobre los archivos permitiendo ver los cambios ocurridos entre las distintas versiones.
	\item Mejorar la seguridad por medio de la búsqueda de virus a los archivos.
\end{itemize}

\subsection*{Lecciones}
\begin{itemize}
	\item Diseñar contenidos cuyos elementos se adapten a la versión del estándar que fue implementada (2004 3ra. edición).
	\item Completar la implementación de SCORM incorporando el <sequencingCollection> según se define en el libro \emph{SCORM Content Aggregation Model (CAM)} \citep{SCORM_CAM}.
\end{itemize}